\begin{frame}[parent={concept:junit}, hasprev=false, hasnext=true]
\frametitle{Assertion}
\label{concept:assertion}
\label{concept:junit-assertion}

\begin{block:concept}{Assertion}
An assertion is a statement that evaluates as true.
\end{block:concept}

\begin{block:fact}{}
\begin{itemize}
	\item Assertions work as oracles: they confront obtained and expected
	outputs, pointing any discrepancies, and enabling the automatic test
	cases execution.

	\item JUnit only records failed assertions.
\end{itemize}
\end{block:fact}

\hfill
\refie{example:junit-raw-assertion}{\beamerbutton{Example: Test case with assertion}}
\end{frame}


\begin{frame}[hasprev=true, hasnext=false]
\frametitle{Assertion}
\framesubtitle{JUnit assertions}

\begin{block:fact}{JUnit assertions}
\begin{itemize}
	\item Instead of using Java's default assertion mechanism, one can use
	assertions provided by JUnit.

	\item JUnit implements several assertions in the class \srccode{Assert}:
	\begin{itemize}
		\item \srccode{assertThat}
		\item \srccode{assertArrayEquals}, \srccode{assertEquals}
		\item \srccode{assertSame}, \srccode{assertNotSame}
		\item \srccode{assertTrue}, \srccode{assertFalse}
		\item \srccode{assertNull}, \srccode{assertNotNull}
		\item \srccode{fail}
	\end{itemize}
\end{itemize}
\end{block:fact}
\end{frame}
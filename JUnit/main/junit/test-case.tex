\begin{frame}[parent={concept:junit}, hasprev=false, hasnext=true]
\frametitle{JUnit test case}
\label{concept:junit-test-case}
\label{concept:junit-testcase}


\begin{block:concept}{Test case}
A test case is a pair consisting of test data (a set of values, one for each
input variable) to be input to the program and the expected output.
\end{block:concept}

\begin{block:concept}{JUnit test case}
A JUnit test case is the implementation of a test case as a Java method
annotated with \srccode{@org.junit.Test}.
\end{block:concept}

\begin{block:fact}{How to define a test case}
\begin{itemize}
	\item In general, each test case is defined in a different method within a
	Java class.

	\item Test methods neither accept parameters nor return a value.
\end{itemize}
\end{block:fact}

\hfill
\refie{example:junit-testcase}{\beamerbutton{Example: JUnit test case}}
\end{frame}



\begin{frame}[hasprev=true, hasnext=true]
\frametitle{JUnit test case}
\framesubtitle{Compilation}
\label{procedure:junit-test-case:compilation}

\begin{block:procedure}{How to compile a test case}
\begin{itemize}
	\item To compile a test case, run the Java compiler against the test case
	file.
	\begin{itemize}
		\item Remember to include the JUnit library in the classpath.
	\end{itemize}
\end{itemize}
\end{block:procedure}

\hfill
\refie{example:junit-testcase-compilation}{\beamerbutton{Example: JUnit test case compilation}}
\end{frame}


\begin{frame}
\frametitle{JUnit test case}
\framesubtitle{Execution}
\label{procedure:junit-test-case:execution}

\begin{block:procedure}{How to run a test case}
\begin{itemize}
	\item To run JUnit test cases from the command line, run
	\srccode{java org.junit.runner.JUnitCore TestClass1 TestClass2}.
\end{itemize}
\end{block:procedure}

\hfill
\refie{example:junit-testcase-execution}{\beamerbutton{Example: JUnit test case execution}}
\end{frame}



\begin{frame}[hasprev=true, hasnext=false]
\frametitle{JUnit test case}
\framesubtitle{Outcomes}
\label{fact:junit-test-case:outcomes}


\begin{block:fact}{Outcomes}
\begin{itemize}
	\item A test case fails when the generated output value is different than
	the expected output value.

	\item A test case succeeds when the generated output value is equal to the
	expected output value.
\end{itemize}
\end{block:fact}


\begin{block:fact}{How does it detects a failures?}
\begin{itemize}
	\item A JUnit test case fails when an assertion fails (when an
	\srccode{AssertionError} exception is thrown by the test case).
\end{itemize}
\end{block:fact}

\hfill
\refie{example:junit-testcase-outcomes}{\beamerbutton{Example: JUnit test case execution outcomes}}
\end{frame}
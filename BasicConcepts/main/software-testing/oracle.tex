\begin{frame}[parent={cmap:software-testing-foundations}, hasprev=false, hasnext=true]
\frametitle{Oracle}

\begin{block:fact}{Is it correct?}
\begin{itemize}
	\item Given a set of input conditions and the observable results of the
	computation, who decides whether the results are correct?

	\item Somebody or something must check whether the software, for a given
	test case, has operated correctly.
\end{itemize}
\end{block:fact}
\end{frame}


\begin{frame}[hasprev=true, hasnext=true]
\frametitle{Oracle}
\label{concept:oracle}

\begin{block:concept}{Definition}
An oracle is any software, process, or data that provides the test designer
with the expected result of a test case.
\end{block:concept}

\begin{block:fact}{}
\begin{itemize}
	\item An oracle decides whether output values are correct against what is
	specified.
	\begin{itemize}
		\item An oracle is required to determine whether a fault was revealed.
	\end{itemize}

	% TODO: Propose a better classification of oracles
	\item Some examples of oracle: human guess (kiddie oracle), regression
	test suite, validated data, purchased test suite, existing software.
\end{itemize}
\end{block:fact}
\end{frame}



\begin{frame}
\label{concept:kiddie-oracle}
\frametitle{Oracle}
\framesubtitle{Kiddie oracle}

\begin{block:concept}{Definition}
A kiddie oracle is obtained from running the software and seeing the output. If
it looks about right, it must be right.
\end{block:concept}

\begin{block:fact}{Why should I use a kiddie oracle?}
\begin{itemize}
	\item Actually, you should not use it, as it is error prone.

	\item Nonetheless, it is better than nothing.
	\begin{itemize}
		\item And, if the expected behavior of the application is not
		documented, it is up to the user to detect the correct result anyway.
	\end{itemize}
\end{itemize}
\end{block:fact}

\hfill
\refie{example:kiddie-oracle}{\beamerbutton{Example: Human guess (kiddie) oracle}}
\end{frame}



\begin{frame}
\label{concept:regression-test-suite-oracle}
\frametitle{Oracle}
\framesubtitle{Regression test suite oracle}

\begin{block:concept}{Definition}
A regression test suite oracle  is obtained from running the test case and
comparing the output to the results of the same test cases run against a
previous version of the software.
\end{block:concept}

\begin{block:fact}{Why should I use a regression suite oracle?}
\begin{itemize}
	\item Regression test ensures that the modified system functions as per
	its specification.

	\item It ensures that old errors will not appear again (or that, at least,
	they will be early detected).
\end{itemize}
\end{block:fact}

\hfill
\refie{example:mozilla-firefox-regression-test-suite-oracle}{\beamerbutton{Example: Regression test suite oracle for Mozilla Firefox}}
\end{frame}



\begin{frame}[hasprev=true, hasnext=false]
\label{concept:purchased-test-suite-oracle}
\frametitle{Oracle}
\framesubtitle{Purchased test suite oracle}

\begin{block:concept}{Definition}
A purchased test suite oracle consists in running the software against a
standardized test suite that has been previously created and validated.
\end{block:concept}

\begin{block:fact}{Why should I use a purchased oracle?}
\begin{itemize}
	\item Usually it is required that a software passes a test suite oracle
	in order assess its compliance to a particular technology or standard.

	\item Sometimes its simply easier to buy a test suite instead of developing
	your own.
\end{itemize}
\end{block:fact}


\hfill
\refie{example:java-test-suite-oracle}{\beamerbutton{Example: Java TCK}}
\end{frame}

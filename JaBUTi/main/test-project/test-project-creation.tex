\begin{frame}[parent={cmap:jabuti-test-project},hasnext=true,hasprev=true,fragile]
\frametitle{Test project}
\framesubtitle{Create new test project}
\label{procedure:create-test-project}

\begin{block:fact}{Test project}
\begin{itemize}
	\item From the generated \srccode{.class} files, the user can create a
	test project using JaBUTi.
\end{itemize}
\end{block:fact}

\begin{block:procedure}{Execute JaBUTi}
\begin{enumerate}
	\item Invoke JaBUTi's graphical interface.
	\begin{itemize}
		\item Double-click \srccode{Jabuti-bin.jar}.

		\item Supposing that JaBUTi is installed on \srccode{/opt/JaBUTi},
		it is possible to start the application from the command line:
		\begin{lstlisting}
			$ java -jar /opt/JaBUTi/Jabuti-bin.jar
		\end{lstlisting}
	\end{itemize}
\end{enumerate}
\end{block:procedure}
\end{frame}



\begin{frame}[fragile]
\frametitle{Test project}
\framesubtitle{Create new test project}
\label{procedure:create-test-project}
\label{concept:base-class}
\label{concept:classpath}

\begin{block:procedure}{Create new project}
\begin{enumerate}
	\item Select a base \srccode{.class} file from \srccode{File/Open Class}
	menu.

	\item Select the directory where the base class file is located, and then
	select the base class file itself.

	\item Check the \srccode{Package} field.
	\begin{itemize}
		\item Once the base class file is selected, the tool automatically
		identifies the package that it belongs to and fills out the
		\srccode{Package} field with the package's name.
	\end{itemize}

	\item Set the \srccode{Classpath} field.
	\begin{itemize}
		\item It should contain only the path necessary to run the selected
		base class.
	\end{itemize}
\end{enumerate}
\end{block:procedure}
\end{frame}


\begin{frame}
\frametitle{Test project}
\framesubtitle{Create new test project}
\label{procedure:create-test-project}
\label{concept:set-of-application-classes}
\label{concept:classes-to-be-ignored}
\label{concept:classes-to-be-tested}

\begin{block:procedure}{Create new project}
\begin{enumerate}
	\setcounter{enumi}{5}
	\item<1-> Click the \srccode{Open} button. The \srccode{ProjectManager}
	windows will be displayed.
	\begin{itemize}
		\item From the selected base class file, the tool identifies the
		complete set of classes necessary to its execution.
	\end{itemize}

	\item<2-> From the \srccode{Project Manager} window (left side), the user
	can 	select the class files that will be tested.
	\begin{itemize}
		\item At least one class file must be selected.

		\item The base class, if it is the driver, should not be selected.
	\end{itemize}
\end{enumerate}
\end{block:procedure}
\end{frame}


\begin{frame}
\frametitle{Test project}
\framesubtitle{Create new test project}
\label{procedure:create-test-project}

\begin{block:procedure}{Create new project}
\begin{enumerate}
	\setcounter{enumi}{7}
	\item<1-> A name must be given to the project being created by clicking on the
	\srccode{Select} button.

	\item<2-> Click the \srccode{Ok} button. After this action, JaBUTi will:
	\begin{itemize}
		\item create a new project,

		\item construct the DUG for each method of each class under
		testing,

		\item derive the complete set of test requirements for each
		criterion,

		\item calculate the weight of each test requirement,

		\item and show the bytecode of a class under testing.
	\end{itemize}
\end{enumerate}
\end{block:procedure}

\end{frame}



\begin{frame}[c, ,hasnext=false]
\frametitle{Test project demonstration}

\insertmovie{resources/JaBUTi/JaBUTi-VendingMachine/JaBUTi-VendingMachine-NewProject/JaBUTi-VendingMachine-NewProject}

\end{frame}

\section{JaBUTi-CLI}


\begin{frame}
\frametitle{JaBUTi-CLI}

\begin{cdt-concept}
JaBUTi-CLI is a set of scripts that provides a command line interface to use
JaBUTi. 
\end{cdt-concept}

\end{frame}



\begin{frame}
\frametitle{JaBUTI-CLI}
\framesubtitle{Configuration}

\begin{cdt-facts}
The configuration of JaBUTi-CLI is defined in a .jbt file. This file is a shell
script that provides the following information as variables:
\begin{itemize}
	\item JABUTI_HOME: Directory where JaBUTi is installed.
	\item ORIG_DIR: Directory with the original classes of the software under
	testing.
	\item TEST_SCRIPT: The build.xml (Ant build file) that contains the targets
	the run the test cases.
	\item TEST_SCRIPT_OUT: Directory where the outputs from the execution of
	the test cases are saved to.
	\item TEST_EXEC_CMD: Tha target that run the test cases.
	\item SPAGO_DIR: Directory where the SPAGO4Q xml file will be saved to.
	\item SPAGO_ID: The project ID, in the SPAGO4Q, of the software under
	testing.
\end{itemize}
\end{cdt-facts}

\begin{cdt-facts}
The configuration file can also define the following (optional) variables:
\begin{itemize}
	\item ORIG_JAR: Jar package with original classes.
	\item INSTRUM_JAR: Jar package name that will be generated by
	instrumentation.
\end{itemize}
\end{cdt-facts}
\end{frame}


\begin{frame}
\frametitle{JaBUTi-CLI}
\framesubtitle{Initialization}

\begin{cdt-concept}
The jabuti-initialize script analyzes the application under testing, then
creates and configures some JaBUTi files and directories that will be used by
the next scripts. 
\end{cdt-concept}

\end{frame}





\begin{frame}
\frametitle{JaBUTi-CLI}
\framesubtitle{Instrumentation}

\begin{cdt-concept}
The jabuti-instrument script is responsible for instrumenting the classes in
\srccode{$ORIG_JAR}.
\end{cdt-concept}

\begin{cdt-facts}
\begin{itemize}
	\item If this variable is not informed, the classes of \srccode{$ORIG_DIR}
	will be instrumented. As a result, a jar package with classes instrumented
	will be saved and named according to the value passed in \srccode{$INSTRUM_JAR}
	variable.
	
	\item If the \srccode{$INSTRUM_JAR} is empty, a temporary file with the
	classes instrumented are generated.
\end{itemize}
\end{cdt-facts}
\end{frame}



\begin{frame}
\frametitle{JaBUTi-CLI}
\framesubtitle{Execution}

\begin{cdt-concept}
The jabuti-execute script is in charge of executing the command informed in the
\srccode{$TEST_EXEC_CMD}, generating trace files in the
\srccode{$TEST_SCRIPT_OUT} directory.  
\end{cdt-concept}
\end{frame}


\begin{frame}
\frametitle{JaBUTi-CLI}
\framesubtitle{Reporting}

\begin{cdt-concept}
The jabuti-consolidate script analyzes the trace files generated during test
suite execution and generates data coverage reports.
\end{cdt-concept}

\begin{cdt-facts}
\begin{itemize}
	\item In the end of process, the data from all these reports are
	consolidated in a SPAGO4Q XML file, using the \srccode{$SPAGO_ID} as then
	project identifier. 
\end{itemize}
\end{cdt-facts}
\end{frame}

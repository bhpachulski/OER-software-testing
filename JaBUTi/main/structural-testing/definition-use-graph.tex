\begin{frame}[parent={cmap:structural-software-testing},hasnext=true,hasprev=true]
\frametitle{Definition-use graph}
\label{concept:program-graph}
\label{concept:definition-use-graph}
\label{concept:dug}

\begin{block:concept}{Definition}
The definition-use graph (DUG) is a program graph suitable to represent the
different status of a variable.
\end{block:concept}

\begin{block:fact}{Definition-use graph and control flow graph}
\begin{itemize}
	\item The definition-use graph is an extension of the control flow graph.

	\item The definition-use graph includes information about variable
	definitions, uses and destructions on each node.
	\begin{itemize}
		\item It contains all definition-use pairs for a program.
	\end{itemize}
\end{itemize}
\end{block:fact}
\end{frame}


\begin{frame}
\frametitle{Definition-use graph}
\framesubtitle{Graph elements}
\label{concept:cfg-graph-elements}

\begin{block:fact}{Graph elements}
\begin{itemize}
	\item The nodes of a program graph are indivisible code blocks.

	\item The edges of a program graph represent the possible transference of
	execution between code blocks.

	\item There is just one input node, which corresponds to the code block
	with the first statement of the program unit.

	\item It is possible to have many output nodes, i.e., nodes without a
	succeeding node.
\end{itemize}
\end{block:fact}

\hfill
\refie{example:program-graph}{\beamerbutton{Example: Program graph}}
\refie{example:identifier-cfg}{\beamerbutton{Example: Control-flow graph for Identifier}}
\end{frame}


\begin{frame}
\frametitle{Definition-use graph}
\framesubtitle{Code block}
\label{concept:statement}
\label{concept:statement-block}
\label{concept:code-block}

\begin{block:concept}{Definition}
A code block of a program is a sequence of consecutive statements with a single
entry point and a single exit point.
\end{block:concept}

\begin{block:fact}{Control flow and code blocks}
\begin{itemize}
	\item Control always enter a code block at its entry point and exits from
	its exit point.

	\item There is no possibility of exit or a halt at any point inside a code
	block except at its exit point.
\end{itemize}
\end{block:fact}
\end{frame}


\begin{frame}
\frametitle{Definition-use graph}
\framesubtitle{Code block}

\begin{block:fact}{Control flow and code blocks}
\begin{itemize}
	\item The entry and exit points of a code block coincide when the code
	block contains only one statement.

	\item Function calls are often treated as code blocks of their own because
	they cause the control to be transfered away from the currently executing
	function and hence raise the possibility of abnormal termination of the
	program.
\end{itemize}
\end{block:fact}
\end{frame}


\begin{frame}
\frametitle{Definition-use graph}
\framesubtitle{Decision}
\label{concept:definition}

\begin{block:concept}{Definition}
A decision is a statement that causes a deviation in the flow of a program.
\end{block:concept}

\begin{block:fact}{Decision statements}
\begin{itemize}
	\item The transference of execution between code blocks is a consequence of
	decision statements.

	\item Most high-level languages provides statements, such as \srccode{if},
	\srccode{while} and \srccode{switch}, to serve as context for decisions.
\end{itemize}
\end{block:fact}
\end{frame}


\begin{frame}
\frametitle{Definition-use graph}
\framesubtitle{Variable definition}
\label{concept:variable-definition}

\begin{block:concept}{Definition}
A variable definition is the assignment of a value before a variable be used.
\begin{itemize}
	\item A variable definition occurs when a value is stored in a memory
	position.
\end{itemize}
\end{block:concept}
\end{frame}


\begin{frame}
\frametitle{Definition-use graph}
\framesubtitle{Variable use}
\label{concept:variable-use}

\begin{block:concept}{Definition}
A variable use is when the reference to such a variable does not
define it.
\end{block:concept}

\begin{block:fact}{Types of variable use}
\begin{itemize}
	\item There are two kinds of variable use: computational use and
	predicative use.
	\begin{itemize}
		\item A variable computational use (c-use) directly affects the
		computation being performed or allows the result from a previous
		variable definition to be observed.

		\item A variable predicative use (p-use) directly affects the control
		flow of the product implementation.
	\end{itemize}
\end{itemize}
\end{block:fact}
\end{frame}


\begin{frame}
\label{concept:dug-construction}
\frametitle{Definition-use graph}
\framesubtitle{Definition-use graph construction}

\begin{block:procedure}{DUG construction (1/2)}
\begin{enumerate}
	\item Consider a control-flow graph $G = (N, E)$ of a program $P$, where
	$N$ is the set of nodes and $E$ the set of edges. Each node in $G$
	corresponds to a code block in $P$: those blocks are denoted as
	$b_1, b_2, ..., b_k$, assuming that $P$ contains $k > 0$ code blocks.

	\item Let $def_i$ denote the set of variables defined in block $i$.

	\item Let $c-use_i$ denote the set of variables that have a computational
	use in block $i$.

	\item Let $p-use_i$ denote the set of variables that have a predicative
	use in block $i$.
\end{enumerate}
\end{block:procedure}

\hfill
\refie{example:identifier-dug}{\beamerbutton{Example: Definition-use graph for Identifier}}
\end{frame}



\begin{frame}
\frametitle{Definition-use graph}
\framesubtitle{Definition-use graph construction}

\begin{block:procedure}{DUG construction (2/2)}
\begin{enumerate}
	\setcounter{enumi}{4}
	\item Compute $def_i$, $c-use_i$, and $p-use_i$ for each code block $i$ in
	$P$.

	\item Associate each node $i$ in $N$ with $def_i$, $c-use_i$, and $p-use_i$.

	\item For each node $i$ that as a nonempty predicative use sets and ends in
	condition $C$, associate edges $(i, j)$ and $(i, k)$ with $C$ and $!C$,
	respectively, given that edge $(i, j)$ is taken when the condition is true
	and $(i, k)$ is taken when the condition is false.
\end{enumerate}
\end{block:procedure}

\hfill
\refie{example:identifier-dug}{\beamerbutton{Example: Definition-use graph for Identifier}}
\end{frame}
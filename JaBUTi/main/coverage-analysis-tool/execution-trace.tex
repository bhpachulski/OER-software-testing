\begin{frame}
\frametitle{Coverage analysis tool}
\framesubtitle{Trace and coverage}

\begin{block:procedure}{Trace collection and coverage analysis}
\begin{enumerate}
	\setcounter{enumi}{10}
	\item The trace file, generated by the instrumented classes, must be
	collected.
	\begin{itemize}
		\item The trace information with respect to the current execution is
		appended in a trace file with the same name of the testing project but
		with the extension \srccode{.trc} instead of \srccode{.jbt}.
	\end{itemize}

	\item Trace information is used to update the coverage of the test set
	with respect to the test criteria supported by JaBUTi.
	\begin{itemize}
		\item Every time the size of the trace file increase, the Update button
		in the JaBUTi's graphical interface becomes red, indicating that the
		coverage information can be updated.
	\end{itemize}
\end{enumerate}
\end{block:procedure}
\end{frame}



\begin{frame}[imacidie]
\frametitle{Coverage analysis tool}
\framesubtitle{Trace and coverage}

\begin{block}{Example}
\insertmovie{resources/JaBUTi/JaBUTi-VendingMachine/JaBUTi-VendingMachine-TraceAndCoverage/JaBUTi-VendingMachine-TraceAndCoverage}
\end{block}
\end{frame}

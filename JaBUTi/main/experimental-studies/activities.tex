\begin{frame}[parent={cmap:experimental-studies},hasnext=true,hasprev=true]
\frametitle{Experimental studies}
\framesubtitle{Experimental study activities}

\begin{block:procedure}{Activities}
\begin{enumerate}
	\item Select and prepare the programs to be tested (the population of the
	experiment).

	\item Select the software testing tools.

	\item Create test cases adequate for a set of test criteria (the test
	criteria are the intervention of the experiment).

	\item Run the program under test using the test cases.

	\item Identify infeasible requirements.

	\item Collect and analyze the results.
\end{enumerate}
\end{block:procedure}
\end{frame}


\begin{frame}
\frametitle{Experimental studies}
\framesubtitle{Program selection}

\begin{block:fact}{Program selection}
\begin{itemize}
	\item Programs should be selected for testing.

	\item Open source software are an interesting choice (as their source
	code is readily available).

	\item Simple programs, such as \srccode{cal} and \srccode{identifier},
	can also be used (but with reduced confidence in the evalution of the
	properties of the test criteria).
\end{itemize}
\end{block:fact}
\end{frame}



\begin{frame}
\frametitle{Experimental studies}
\framesubtitle{Selection of Software testing tools}

\begin{block:fact}{Test tools selection}
\begin{itemize}
	\item It is a requirement that the test be automated as much as possible.

	\item A common requirement is that the test tools provide a script
	execution mode (so that the tester intervention is not required to run
	every aspect of the tool).

	\item JaBUTi can be run from the command line, using scripts.
\end{itemize}
\end{block:fact}
\end{frame}



\begin{frame}
\frametitle{Experimental studies}
\framesubtitle{Test cases generation}

\begin{block:fact}{Test cases generation}
\begin{itemize}
	\item Test cases are usually generated randomly.
	\begin{itemize}
		\item It is easy to generate input data randomly.
		\item It eliminates the influence of the tester on the generated data
		(and any threat of validity due to the tester influence).
	\end{itemize}

	\item If the required test coverage is not achieved, test cases should
	be manually and added to the test set.
\end{itemize}
\end{block:fact}
\end{frame}



\begin{frame}
\frametitle{Experimental studies}
\framesubtitle{Execution}

\begin{block:fact}{Program execution}
\begin{itemize}
	\item Using the test tool, the program is run using the input data from
	the test cases.

	\item The output data of the program is compared with the expected output
	data (as of the test cases).

	\item Test requirements satisfaction is analyzed using the program results
	and execution trace.
\end{itemize}
\end{block:fact}

\begin{block:fact}{Infeasible requirements identification}
\begin{itemize}
	\item Infeasible requirements (non-executable paths, equivalent mutants)
	should be identified.

	\item Test set should be augmented until a $C-adequate$ test set is
	defined.
\end{itemize}
\end{block:fact}
\end{frame}


\begin{frame}[hasnext=false]
\frametitle{Experimental studies}
\framesubtitle{Execution}
\label{concept:experimental-studies:analysis}

\begin{block:fact}{Analysis}
\begin{itemize}
	\item Test results -- cost, efficacy and strength -- are calculated and
	compared.

	\item Relations between test criteria are established.
\end{itemize}
\end{block:fact}

\hfill
\refie{example:subsume-relation}{Example: subsume relation}
\end{frame}
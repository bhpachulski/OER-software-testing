\begin{frame}[hasprev=false, hasnext=true]
\label{example:la-tosca-at-san-diego}
\frametitle{La Tosca at San Diego}
\framesubtitle{Description}

\begin{itemize}
	\item The opera La Tosca (Giacomo Puccini) debuted just over one hundred
	years ago, at the Teatro Constanzi in Rome on January 14, 1900.

	\item Soon after its premiere, it became one of the most popular operas in
	the repertoire, and it remains so to this day.

	\item It was the candelabra that played a prominent role in a San Diego
	performance of Tosca in 1956.
	\begin{itemize}
		\item The script called for Tosca to blow out the four candles in the
		candelabra before dramatically placing a candle on either side of
		Scarpia and a crucifix on his breast and exiting the stage.
	\end{itemize}
\end{itemize}
\end{frame}


\begin{frame}[hasprev=true, hasnext=false]
\frametitle{La Tosca at San Diego}
\framesubtitle{Problem}

\begin{itemize}
	\item In San Diego, the candles were electric, and the order of their
	going out was fixed on a computer tape along with all the rest of the
	lighting cues.

	\item The tape obeyed the stage manager's signal and snuffed the
	candles exactly as Tosca blew them out.
	\begin{itemize}
		\item Except that, on this occasion, the programming was wrong and
		it blew them out in a different order from hers.

		\item She blew to the right, the candle on the left went out, she
		blew the back one, the one in front went out!
	\end{itemize}
\end{itemize}
\end{frame}

\begin{frame}[hasprev=false, hasnext=true]
\frametitle{Equivalence partition example}

\begin{block:fact}{DIMENSION}
Assume that you are developing a compiler for a subset of an ancient language
called Fortran and that your team is currently implementing the statement
\srccode{DIMENSION}. The following requirements must be satisfied:

\begin{enumerate}
	\item The syntax of the \srccode{DIMENSION} statement is
	\texttt{DIMENSION ad[, ad]*}, where:
	\begin{itemize}
		\item \srccode{ad} is an array descriptor of the form
		\texttt[n(d[ ,d]*)},

		\item \srccode{n} is the symbolic name of the array,

		\item \srccode{d} is the dimension declarator.
	\end{itemize}

	\item Symbolic names can be one to six letters or digits, the first of
	which must be a letter.
\end{enumerate}
\end{block:fact}
\end{frame}


\begin{frame}[hasprev=false, hasnext=true]
\frametitle{Equivalence partition example}

\begin{block:fact}{DIMENSION}
\begin{enumerate}
	\item The minimum and maximum numbers of dimension declarations that can
	be specified for an array are one an seven, respectively.

	\item The syntax of a dimension declarator is \srccode{lb: ]ub}, where
	\srccode{lb} and \srccode{ub} are, respectively, the lower and upper
	dimension bounds.

	\item If no \srccode{lb} is specified, it is assumed to be one.

	\item The value of \srccode{ub} must be greater or equal to \srccode{lb}.

	\item The \srccode{DIMENSION} statement may be continued over multiple
	lines.
\end{enumerate}
\end{block:fact}
\end{frame}


\begin{frame}[hasprev=true, hasnext=true]
\frametitle{Equivalence partition example}
\framesubtitle{Classes}

\begin{block}{Equivalence classes form}
\begin{tabularx}{\textwidth}{|X|X|}
\textbf{Input condition}		& \textbf{Valid class}\\\hline
Number of array descriptors 	& one (1), more than one (2)\\\hline
Size of the array name			& 1-6 (4)\\\hline
Array name						& has letters (7), has digits (8)\\\hline
Array name starts with letter	& yes (10)\\\hline
Number of dimensions			& 1-7 (12)\\\hline
Lower bound is specified		& no, yes and it is a number (15)\\\hline
Upper bound is specified		& no;  yes and it is equal or greater than lower bound (17)\\\hline
Statement has multiple lines	& yes (21), no (22)\\
\end{tabularx}
\end{block}
\end{frame}




\begin{frame}[hasprev=true, hasnext=true]
\frametitle{Equivalence partition example}
\framesubtitle{Classes}

\begin{block}{Equivalence classes form}
\begin{tabularx}{\textwidth}{|X|X|}
\textbf{Input condition}		& \textbf{Invalid class}\\\hline
Number of array descriptors 	&  none (3) \\\hline
Size of the array name			&  0 (5), more than 6 (6)\\\hline
Array name						&  has something else (9)\\\hline
Array name starts with letter	&  no (11)\\\hline
Number of dimensions			&  0 (13), more than 7 (14)\\\hline
Lower bound is specified		&  yes, but it is not a number (16)\\\hline
Upper bound is specified		&  yes, but it is not a number (18); yes, but it is less than lower bound (19); yes, but there is not lower bound (20)\\\hline
Statement has multiple lines	&  -\\
\end{tabularx}
\end{block}
\end{frame}




\begin{frame}
\frametitle{Equivalence partition example}
\framesubtitle{Test cases}

\begin{block}{Test cases and classes}
\begin{itemize}
	\item For each test case, the test requirements (equivalence classes) that
	they cover are in bold face.
	\begin{itemize}
		\item One single test case can cover several valid classes.

		\item There must have one test case for each invalid class.
	\end{itemize}
\end{itemize}
\end{block}


\begin{block}{Test cases for valid classes}
\begin{itemize}
	\item \srccode{DIMENSION A(2)} \textbf{(1, 4, 7, 22)}
	\item \srccode{DIMENSION A 12345(9, 100), BBB(-15:30)} \textbf{(2, 8, 15, 17, 21)}
\end{itemize}
\end{block}
\end{frame}


\begin{frame}[hasprev=true, hasnext=false]
\frametitle{Equivalence partition example}
\framesubtitle{Test cases}

\begin{block}{Test cases for invalid classes}
\begin{itemize}
	\item \srccode{DIMENSION} \textbf{(3)}
	\item \srccode{DIMENSION (10)} \textbf{(5)}
	\item \srccode{DIMENSION A234567(2)} \textbf{(6)}
	\item \srccode{DIMENSION A.1(2)} \textbf{(9)}
	\item \srccode{DIMENSION 1A(10)} \textbf{(11)}
	\item \srccode{DIMENSION B} \textbf{(13)}
	\item \srccode{DIMENSION B(1, 2, 3, 4, 5, 6, 7, 8)} \textbf{(14)}
	\item \srccode{DIMENSION B(ABC)} \textbf{(16)}
	\item \srccode{DIMENSION C(4:ABC)} \textbf{(18)}
	\item \srccode{DIMENSION C(4:3)} \textbf{(19)}
	\item \srccode{DIMENSION D(:1)} \textbf{(20)}
\end{itemize}
\end{block}
\end{frame}
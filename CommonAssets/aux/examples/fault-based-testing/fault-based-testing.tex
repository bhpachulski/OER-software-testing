\begin{frame}[hasprev=false, hasnext=true, fragile]
\frametitle{Fault-based testing example}
\label{example:fault-based-testing}

\begin{block}{Example}
Consider the following source code snippet \cite[p. 45]{myers:2004}.
\end{block}

\begin{block}{Source code}
\begin{lstlisting}[language=java]
public void foo(int a, int b, int x) {
	if (a > 1 && b == 0) {
		x = x/a;
	}
	if (a == 2 || x > 1) {
		x = x + 1;
	}
}
\end{lstlisting}
\end{block}
\end{frame}


\begin{frame}[hasprev=true, hasnext=false, fragile]
\frametitle{Fault-based testing}

\begin{columns}[t]
\column{.5\textwidth}
\begin{block}{Mutation analysis}
\begin{itemize}
	\item Mutation analysis performs single changes in the source code.

	\item For example, comparators such as \srccode{==} can be changed to
	\srccode{!=}.

	\item The following test case will reveal the error in the mutant:
	\begin{itemize}
		\item $a = 3, b = 0, x = 3$
	\end{itemize}
\end{itemize}
\end{block}
\column{.5\textwidth}
\begin{block}{Original application}
\begin{lstlisting}[language=java]
public int foo(int a, int b, int x) {
	if (a > 1 && b == 0) {
		x = x/a;
	}
	if (a == 2 || x > 1) {
		x = x + 1;
	}
	return x;
}
\end{lstlisting}
\end{block}
\begin{block}{Mutant}
\begin{lstlisting}[language=java]
public int foo(int a, int b, int x) {
	if (a > 1 && b != 0) {
		x = x/a;
	}
	if (a == 2 || x > 1) {
		x = x + 1;
	}
	return x;
}
\end{lstlisting}
\end{block}

\end{columns}
\end{frame}
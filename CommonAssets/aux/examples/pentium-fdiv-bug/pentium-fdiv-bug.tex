\begin{frame}[hasprev=false, hasnext=true]
\label{example:pentium-fdiv-bug}
\frametitle{Pentium FDIV bug}
\framesubtitle{Description}

\begin{itemize}
	\item In 1994, Intel introduced its Pentium microprocessor, and a
	few months later, a mathematician found that the chip gave incorrect
	answers to certain floating-point division calculations.
	\begin{itemize}
		\item The chip was slightly inaccurate for a few pairs of numbers.
	\end{itemize}

	\item Example: A number multiplied and then divided by the same number
	should result in the original number)
	\begin{itemize}
		\item Expected result: $4195835*3145727/3145727  = 4195835$
		\item Flawed Pentium result: $4195835*3145727/3145727  = 4195579$
	\end{itemize}
\end{itemize}
\end{frame}


\begin{frame}[hasprev=true, hasnext=true]
\frametitle{Pentium FDIV bug}
\framesubtitle{Diagnostic}

\begin{itemize}
	\item The fault was the omission of five entries in a table of 1,066 values
	(part of the chip's circuitry) used by a division algorithm.
	\begin{itemize}
		\item The five entries should have contained the constant +2, but the
		entries were not initialized and contained zero instead.
	\end{itemize}
\end{itemize}
\end{frame}


\begin{frame}[hasprev=true, hasnext=false]
\frametitle{Pentium FDIV bug}
\framesubtitle{Solution}

\begin{itemize}
	\item The mistake that caused the fault was very difficult to find during
	system testing.
	\begin{itemize}
		\item Intel claimed to have run millions of test cases using this
		table.
	\end{itemize}

	\item The table entries were left empty because a loop termination
	condition was incorrect
	\begin{itemize}
		\item The loop stopped storing numbers before it was finished.
	\end{itemize}

	\item This turns out to be a very simple fault to find during unit testing
	\begin{itemize}
		\item Analysis showed that almost any unit level coverage criterion
		would have found this multi-million dollar mistake.
	\end{itemize}
\end{itemize}
\end{frame}
\begin{frame}[parent={concept:mutation-testing}, hasprev=false, hasnext=true]
\frametitle{Mutation testing}
\framesubtitle{Mutation operator}
\label{concept:mutation-operator}

\begin{block:concept}{Mutation operator}
Mutation operators are rules that define the changes to be carried out in a
product P in order to create a mutant.
\end{block:concept}

\begin{block:fact}{}
\begin{itemize}
	\item Mutation operators model the most frequent faults or syntactic
	deviations related to a given programming language.
\end{itemize}
\end{block:fact}
\end{frame}


\begin{frame}[hasprev=true, hasnext=true]
\frametitle{Mutation testing}
\framesubtitle{Mutation operator}

\begin{block:fact}{Objectives}
\begin{itemize}
	\item Mutation operators are designed for a target language and should
	fulfill one of the following objectives:
	\begin{itemize}
		\item Create a simple syntactic change based on typical errors made by
		programmers.

		\item Force test cases to have a desired property (covering a given
		branch in the program, for instance).
	\end{itemize}
\end{itemize}
\end{block:fact}
\end{frame}


\begin{frame}[hasprev=true, hasnext=true]
\frametitle{Mutation testing}
\framesubtitle{Mutation operator}

\begin{block:fact}{Test strategy}
\begin{itemize}
	\item Mutation operators can be selected according to the classes or faults
	to be addressed, allowing the creation of mutants to be done stepwise or
	even to be divided between testers working independently.
\end{itemize}
\end{block:fact}
\end{frame}
\begin{frame}[parent={concept:mutation-testing}, hasprev=false, hasnext=true]
\frametitle{Mutation testing}
\framesubtitle{Mutant}
\label{concept:mutant}

\begin{block:concept}{Mutant}
% TODO: add mutant definition.
\end{block:concept}


\begin{block:fact}{Mutation testing}
\begin{itemize}
	\item In mutation testing, a single mutation is applied to the program P
	under testing.
	\begin{itemize}
		\item Each mutant has a single syntactic transformation related to the
		original program.
	\end{itemize}

	\item Tester has to create test cases that show that mutations create
	incorrect products.
\end{itemize}
\end{block:fact}
\end{frame}



\begin{frame}
\frametitle{Mutation testing}
\framesubtitle{Live mutant}
\label{concept:live-mutant}

\begin{block:concept}{Live mutant}
A live mutant is a mutant that has the same behavior of the product P for each
test case T.
\end{block:concept}

\begin{block:fact}{How to identify one?}
\begin{itemize}
	\item A live mutant must be analyzed by the tester to check whether it is
	equivalent to product P or whether it can be killed by a new test case,
	thus promoting the improvement of the test set T.
\end{itemize}
\end{block:fact}
\end{frame}


\begin{frame}
\frametitle{Mutation testing}
\framesubtitle{Dead mutant}
\label{concept:dead-mutant}

\begin{block:concept}{Dead mutant}
A dead mutant is a mutant that has a diverse behavior from product P on at
least one test case T.
\end{block:concept}
\end{frame}


\begin{frame}
\frametitle{Mutation testing}
\framesubtitle{Equivalent mutant}
\label{concept:equivalent-mutant}

\begin{block:concept}{Equivalent mutant}
A mutant M is said equivalent to a product P if, for all input data $d \in D$,
$M(d) = P(d)$.
\end{block:concept}

\begin{block:fact}{}
\begin{itemize}
	\item An equivalent mutant is created by a mutation operation that does not
	result in a behavioral change for a product P, and, for every test datum in
	the input domain, P and the mutant always compute the same results.
\end{itemize}
\end{block:fact}

\end{frame}


\begin{frame}
\frametitle{Mutation testing}
\framesubtitle{Fault-revealing mutant}
\label{concept:fault-revealing-mutant}

\begin{block:concept}{Fault-revealing mutant}
A mutant M is a fault-revealing mutant to a product P if for any test case $t$
such that $P(t) \neq M(t)$ we can conclude that $P(t)$ is not in accordance
with the expected result
\end{block:concept}

\begin{block:fact}{}
\begin{itemize}
	\item The presence of a fault is revealed by killing the mutant.
\end{itemize}
\end{block:fact}
\end{frame}



\begin{frame}[hasprev=true, hasnext=false]
\frametitle{Mutation testing}
\framesubtitle{Mutation score}
\label{concept:mutation-score}

\begin{block:concept}{Mutation score}
Mutation score is the relation between the number of mutants killed by the
test set T and the difference between the total number of mutants and the
number of mutants equivalent to P.
\end{block:concept}

\begin{block:fact}{Meaning}
\begin{itemize}
	\item Mutation score is an objective measure to evaluate the test set
	adequacy against mutation testing.
	\begin{itemize}
		\item Mutation score ranges from 0 to 1.

		\item The higher the mutation score, the more adequate is the test set.
	\end{itemize}
\end{itemize}
\end{block:fact}
\end{frame}

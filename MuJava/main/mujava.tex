\begin{frame}[parent={cmap:mujava}, hasprev=false, hasnext=true]
\frametitle{muJava}
\label{concept:mujava}

\begin{block:concept}{muJava}
muJava ($\mu{}Java$) is a mutation tool for Java programs that automatically
generates mutants using mutation analysis and interface mutation criteria.
\end{block:concept}

\begin{block:fact}{General information}
\begin{itemize}
	\item It was developed by Yu Seung Ma, Yong Rae Kwon and Jeff Offutt.

	\item Main page for muJava:
	\begin{itemize}
		\item \url{http://cs.gmu.edu/~offutt/mujava/}
	\end{itemize}

	\item A Eclipse plugin is also available (developed by Laurie Willians and
	Ben Smith):
	\begin{itemize}
		\item \url{http://muclipse.sourceforge.net/}
	\end{itemize}

	\item Compatible with Java 1.5 and 1.6 (but it still does not support
	generics).
\end{itemize}
\end{block:fact}
\end{frame}


\begin{frame}[hasprev=true, hasnext=true]
\frametitle{muJava}
\framesubtitle{Mutation operators}

\begin{block:fact}{Method level operators}
\begin{itemize}
	\item
\end{itemize}
\end{block:fact}


\begin{block:fact}{Class level operators}
\begin{itemize}
	\item Set of 24 operators that are specialized to object-oriented faults.
\end{itemize}
\end{block:fact}
\end{frame}


\begin{frame}[hasprev=true, hasnext=false]
\frametitle{muJava}
\framesubtitle{Limitations}

\begin{block:fact}{Limitations}
\begin{itemize}
	\item It does not implement any automatic equivalent mutant detection
	technique.
	\begin{itemize}
		\item Equivalents mutants must be identified by hand.
	\end{itemize}
\end{itemize}
\end{block:fact}
\end{frame}


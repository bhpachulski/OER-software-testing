\begin{frame}[parent={concept:mujava}, hasprev=false, hasnext=true]
\frametitle{MuClipse}

\begin{block:concept}{MuClipse}
MuClipse is an Eclipse Plugin which provides a bridge between the existing
MuJava API and the Eclipse Workbench.
\end{block:concept}

\begin{block:fact}{Benefits}
\begin{itemize}
	\item Most of the configuration (source files, test files) are read from
	Eclipse's project definition.

	\item JUnit support.
\end{itemize}
\end{block:fact}

\begin{block:fact}{Generation information}
\begin{itemize}
	\item Developed by B. Smith and L. Williams.

	\item Main site:
	\begin{itemize}
		\item \url{http://muclipse.sourceforge.net/}
	\end{itemize}
\end{itemize}
\end{block:fact}
\end{frame}


\begin{frame}[hasprev=true, hasnext=true]
\frametitle{MuClipse}
\framesubtitle{Installation}

\begin{block:procedure}{Installation}
\begin{enumerate}
	\item Add a new update site to Eclipse:
	\begin{itemize}
		\item \url{http://muclipse.sf.net/site}
	\end{itemize}

	\item Select the update site just created.

	\item Select the MuClipse feature and install it.

	\item Restart the workbench after the installation.
\end{enumerate}
\end{block:procedure}
\end{frame}


\begin{frame}
\frametitle{MuClipse}
\framesubtitle{Project configuration}

\begin{block:procedure}{Project configuration}
\begin{enumerate}
	\item Change all test case's methods annotated with \srccode{@Before} and
	\srccode{@After} to public.

	\item Check if you are using a Java runtime environment version 1.6 or
	greater in the build configuration of your project.

	\item Add the package \textbf{extendedOj} to the build path.
	\begin{itemize}
		\item \url{http://muclipse.sourceforge.net/site/extendedOJ.jar}
	\end{itemize}
\end{enumerate}
\end{block:procedure}
\end{frame}



\begin{frame}
\frametitle{MuClipse}
\framesubtitle{Mutant generation}

\begin{block:procedure}{Mutant generation}
\begin{enumerate}
	\item Select the \srccode{Run as} operation in Eclipse and create a new
	running configuration for \srccode{MuClipse: Mutants}.

	\item Configure the directories.
	\begin{itemize}
		\item It is highly recommended to leave the configuration to its
		default.

		\item So, you must change your project configuration to output the
		classes to a single directory (\srccode{bin}).

		\item Type the name of class you want to test in the field
		\srccode{Class to Mutate}.
	\end{itemize}

	\item Select the mutation operators to be used.

	\item Execute the new run configuration!
	\begin{itemize}
		\item If you get an \srccode{error=12, Cannot allocate memory}
		exception and you are using Linux, run the following command in the
		system shell: \srccode{echo 0 > /proc/sys/vm/overcommit_memory}.
	\end{itemize}
\end{enumerate}
\end{block:procedure}
\end{frame}



\begin{frame}
\frametitle{MuClipse}
\framesubtitle{Test execution}

\begin{block:procedure}{Test execution}
\begin{enumerate}
	\item Select the \srccode{Run as} operation in Eclipse and create a new
	running configuration for \srccode{MuClipse: Tests}.

	\item Configure the directories.
	\begin{itemize}
		\item Set the name of the directory where the test cases (their
		source code) are stored.

		\item Select the target class (the class for which you generated the
		mutants).

		\item Select the test case (JUnit class) to be executed.
	\end{itemize}

	\item Configure the testing options.
	\begin{itemize}
		\item If no mutant was generated for a given mutant operator type
		(method or class), unselect it (otherwise the execution will not work
		correctly).
	\end{itemize}

	\item Execute the new run configuration.
\end{enumerate}
\end{block:procedure}
\end{frame}



\begin{frame}
\frametitle{MuClipse}
\framesubtitle{View test results}

\begin{block:procedure}{View test results}
\begin{enumerate}
	\item Select the \srccode{Mutants and Results} view at Windows / Show view /
	Other menu.

	\item Click on the right yellow arrow on the right to reload the latest
	results of test case execution.
\end{enumerate}
\end{block:procedure}
\end{frame}

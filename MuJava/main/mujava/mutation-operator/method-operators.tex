\begin{frame}[parent={concept:mujava}, hasprev=false, hasnext=true]
\frametitle{Method operators}

\begin{block:concept}{Method operators}
Method-level mutation operators apply simple syntactical modifications to the
methods of the classes of the application under testing.
\end{block:concept}

\begin{block:fact}{Design rationale}
\begin{itemize}
	\item Method operators were designed using the selective approach.
	\begin{itemize}
		\item The selection results found that operand and statement
		modifications have poor effectiveness.

		\item Thus, only operators that replace, delete, or insert elements in
		expressions were selected.
	\end{itemize}
\end{itemize}
\end{block:fact}
\end{frame}



\begin{frame}[hasprev=true, hasnext=false]
\frametitle{Method operators}

\begin{block:fact}{Types of expression operators}
\begin{itemize}
	\item Arithmetic operator.
	\item Relational operator.
	\item Conditional operator.
	\item Shift operator.
	\item Logical operator.
	\item Assignments.
\end{itemize}
\end{block:fact}
\end{frame}


\begin{frame}[hasprev=true, hasnext=false]
\frametitle{Method operators}

\begin{block:fact}{Operators}
\begin{itemize}
	\item For arithmetic, conditional and logical operators, it is defined
	replacement, insertion, and deletion mutation operators.
	\begin{itemize}
		\item For arithmetic operators, more operators are defined according
		to the type and number of operands.
	\end{itemize}

	\item For relational, shift, and assignment operators, it is defined
	replacement operators.
\end{itemize}
\end{block:fact}
\end{frame}